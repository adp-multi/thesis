\usepackage{tikz}
\usetikzlibrary{positioning,fit,chains,calc}

\usepackage{gnuplot-lua-tikz}

\makeatletter
% tikz-qtree workaround http://tex.stackexchange.com/a/66927/6255
\let\orig@tikz@edge@to@parent@path=\tikz@edge@to@parent@path
\tikzset{
  original tree type/.code={%
    \let\tikz@edge@to@parent@path=\orig@tikz@edge@to@parent@path
  }
}
\makeatother

\tikzset{
	math mode/.style={
		execute at begin node=$,
		execute at end node=$
	}
}

\tikzset{
	grammar rule tree/.style={
	  math mode,
		baseline={([yshift=-2.3ex]current bounding box.north)}
	}
}

% see http://tex.stackexchange.com/a/88548/6255
\newbox\collectbox
\tikzset{
    raise text/.style={
        execute at begin node={\setbox\collectbox=\hbox\bgroup},
        execute at end node={%
            \egroup%
            \phantom{\copy\collectbox}%
            \kern-\wd\collectbox%
            \smash{\raisebox{#1}{\box\collectbox}}}
    }
}

% styles for RNA grammar diagrams
\tikzset{
	rna grammar rule/.style={
		remember picture, % needed when reverse clipping is used
		start chain,
		node distance=0.25,
		every join/.style={thick},
		basepair edge/.style={
			ultra thick,
			out=50,in=130
		},
		font=\footnotesize
	},
	char/.style={
		on chain,
		inner sep=0pt
	},
	nonterminal/.style={
		rectangle,
		draw,
		on chain,
		minimum width=50pt,
		minimum height=14pt,
	},
	phantom rectangle/.style={
		on chain,
		minimum size=14pt,
	},
	terminal/.style={
		char,
		draw,
		circle,
		minimum size=14pt,
		text height=1.5ex,text depth=.25ex % adjust baselines
	},
	empty word/.style={
		rectangle,
		dashed,
		draw,
		on chain,
		minimum size=14pt
	},
	base/.style={
		circle,
		draw,
		fill,		
		on chain,
		minimum size=14pt
	},
	derive sign/.style={
		char,
		inner xsep=5pt,
		font=\Huge,
		execute at begin node=\raisebox{-0.4cm}{$\rightarrow$}
	},
	or sign/.style={
		char,
		font=\Huge,
		execute at begin node=|
	}
}

% styles for RNA secondary structures
\tikzset{
	rna secondary structure/.style={
		node distance=0.2,
		font=\footnotesize,
		every join/.style={thick},
		basepair edge/.style={ultra thick},
		natural/.style={
			every node/.style={terminal,join},
			start chain
		},
		linear/.style={
			every node/.style={terminal,join},
			start chain,
			basepair edge/.style={
				ultra thick,
				out=90,in=90
			}
		},
		circular/.style 2 args={% number of nodes, radius
			every node/.style={terminal,join},
			start chain=placed {at=(270-\tikzchaincount*360/##1+180/##1:##2)}
		},
		filled bases/.style={
			every node/.style={base,join},
		},
		count bases/.style={
			execute at end node=\tikzchaincount
		},
		dot bracket/.style={% useful for aligning under linear style
			every node/.style={% same as terminal, but without ``draw'' and bigger font
				char,
				circle,
				font=\normalsize\ttfamily,
				minimum size=14pt,
				text height=1.5ex,text depth=.25ex % adjust baselines
			},
			start chain
		},
		dot bracket primary/.style={% useful for aligning under linear style
			dot bracket,
			every node/.style={% same as terminal, but without ``draw'' and bigger font
				char,
				circle,
				font=\footnotesize,
				minimum size=14pt,
				text height=1.5ex,text depth=.25ex % adjust baselines
			},
		}
	}
}

% two dimensional nonterminal
% the phantom nodes should be rectangles (to access their anchors)
\newcommand*{\nonterminaltwo}[4][0.0]{% looseness adjustment (optional),1st node, 2nd node, label text
	\pgfmathsetmacro{\loosea}{0.85+(#1)}
	\pgfmathsetmacro{\looseb}{1.0+(#1)}
	\draw[save path=\myclippath]
				(#2.south east) to [in=90, out=90, looseness=\loosea] node[above=-1pt]{#4} (#3.south west)
				-- (#3.south east)
				to [in=90, out=90, looseness=\looseb] (#2.south west) -- cycle;
	\reverseclip{\myclippath}
}

\tikzset{
  reverseclip/.style={
    clip even odd rule,
    insert path={(current page.north east) --
      (current page.south east) --
      (current page.south west) --
      (current page.north west) --
      (current page.north east)}
  },
  clip even odd rule/.code={\pgfseteorule}
}

\newcommand*{\reverseclip}[1]{
  \begin{pgfinterruptboundingbox}
      \clip[reverseclip] \againpath #1;
  \end{pgfinterruptboundingbox}
}

% http://tex.stackexchange.com/a/28537/6255
% usage: \draw[red] \convexpath{a,b,c,d}{1cm};
% ACHTUNG: nodes müssen im uhrzeigersinn angegeben werden
\newcommand{\convexpath}[2]{
  [   
  create hullcoords/.code={
    \global\edef\namelist{#1}
    \foreach [count=\counter] \nodename in \namelist {
      \global\edef\numberofnodes{\counter}
      \coordinate (hullcoord\counter) at (\nodename);
    }
    \coordinate (hullcoord0) at (hullcoord\numberofnodes);
    \pgfmathtruncatemacro\lastnumber{\numberofnodes+1}
    \coordinate (hullcoord\lastnumber) at (hullcoord1);
  },
  create hullcoords
  ]
  ($(hullcoord1)!#2!-90:(hullcoord0)$)
  \foreach [
  evaluate=\currentnode as \previousnode using \currentnode-1,
  evaluate=\currentnode as \nextnode using \currentnode+1
  ] \currentnode in {1,...,\numberofnodes} {
    let \p1 = ($(hullcoord\currentnode) - (hullcoord\previousnode)$),
    \n1 = {atan2(\x1,\y1) + 90},
    \p2 = ($(hullcoord\nextnode) - (hullcoord\currentnode)$),
    \n2 = {atan2(\x2,\y2) + 90},
    \n{delta} = {Mod(\n2-\n1,360) - 360}
    in 
    {arc [start angle=\n1, delta angle=\n{delta}, radius=#2]}
    -- ($(hullcoord\nextnode)!#2!-90:(hullcoord\currentnode)$) 
  }
}

%\newcommand*{\arcthickness}{1}
%\newcommand*{\myarc}[4]{ % x,y,radius,text  
%
   %node is positioned by splitting the arc in two parts (workaround until new pgf is released)
   %see http://tex.stackexchange.com/a/76369/6255
  %\draw[thick,save path=\arcclippath] 
              %(#1,#2) arc (180:90:#3)
              %node[below=-1.5pt] {\tiny #4}
              %arc (90:0:#3)
              %-- ++(-\arcthickness,0) arc (0:180:#3-\arcthickness) -- cycle;
	%\reverseclip{\arcclippath}
%}
%
%\newcommand{\secstruc}[3]{ % x,y,radius
	%\draw [thick] (#1,#2) -- ++(#3*2,0);
	%\draw [thick,dash pattern=on 2pt off 1pt] (#1+2*#3,#2) arc (0:180:#3);
%}

%\begin{tikzpicture}[remember picture,x=0.3cm,y=0.3cm]
	%
	%\myarc{4.8}{0}{4.1}{Hs}
	%\myarc{2.4}{0}{4.1}{Hs}
	%\myarc{0}{0}{4.1}{G}
	%\secstruc{1.2}{0}{0.5}
	%\secstruc{3.6}{0}{0.5}
	%\secstruc{6}{0}{0.5}
	%\secstruc{10.8}{0}{0.5}
	%
%\end{tikzpicture}


% need matrix functionality for that
%\begin{tikzpicture}[rna secondary structure]
	%\begin{scope}[natural]
		%\node[continue chain=going above] {C};
		%\node {A};
		%\node {A};
		%\node {U};
		%\node {U};
		%\node {U};
		%\node {U};
		%\node[yshift=19pt] {C};
		%\node[continue chain=going right,xshift=20pt] {U};
		%\node[continue chain=going below] {G};
		%\node[] {A};
		%\node[join=with chain-6 by {basepair edge}] {A};
		%\node[join=with chain-5 by {basepair edge}] {A};
		%\node[join=with chain-4 by {basepair edge}] {A};
		%\node[join=with chain-3 by {basepair edge}] {U};
		%\node[join=with chain-2 by {basepair edge}] {U};
		%\node[continue chain=going right,xshift=20pt] {U};
		%\node[continue chain=going above,yshift=80pt,join=with chain-11 by {basepair edge}] {U};
		%\node[join=with chain-10 by {basepair edge}] {C};
		%\node[join=with chain-9 by {basepair edge}] {A};
		%\node {C};
	%\end{scope}
%\end{tikzpicture}

%\begin{tikzpicture}[rna secondary structure]
	%\begin{scope}[circular={21}{2.5}]
		%\node {C};
		%\node {A};
		%\node {A};
		%\node {U};
		%\node {U};
		%\node {U};
		%\node {U};
		%\node {C};
		%\node {U};
		%\node {G};
		%\node {A};
		%\node[join=with chain-6 by {basepair edge}] {A};
		%\node[join=with chain-5 by {basepair edge}] {A};
		%\node[join=with chain-4 by {basepair edge}] {A};
		%\node[join=with chain-3 by {basepair edge}] {U};
		%\node[join=with chain-2 by {basepair edge}] {U};
		%\node {U};
		%\node[join=with chain-11 by {basepair edge}] {U};
		%\node[join=with chain-10 by {basepair edge}] {C};
		%\node[join=with chain-9 by {basepair edge}] {A};
		%\node {C};
	%\end{scope}
%\end{tikzpicture}


	% grammar rules
	%\begin{align*}
		%Z &\rightarrow
		%\begin{tikzpicture}[grammar rule tree]
			%\node {\op{f}_1}
				%child { node {X} }
				%child { node {X} }
				%child { node {a} };
		%\end{tikzpicture} \mid 
		%\begin{tikzpicture}[grammar rule tree]
			%\node {\op{f}_2}
				%child { node {b} }
				%child { node {X} };
		%\end{tikzpicture}
		%\\
		%X &\rightarrow 
		%\begin{tikzpicture}[grammar rule tree]
			%\node {\op{f}_3}
				%child { node {b} }
				%child { node {b} };
		%\end{tikzpicture} \mid 
		%\begin{tikzpicture}[grammar rule tree]
			%\node {\op{f}_3}
				%child { node {c} }
				%child { node {c} };
		%\end{tikzpicture}
	%\end{align*}