\usepackage{amsmath}
\usepackage{amsthm}
\usepackage{thmtools} % for custom 'qed' symbol outside proofs
\usepackage{amssymb}
\usepackage{mathtools} % for math versions of \llap, \rlap etc.

\usepackage[framemethod=TikZ]{mdframed}
\colorlet{shadecolor}{gray!10}

\usepackage{bm} % bold math: \bm{..}
\newcommand{\bftt}[1]{\textbf{\texttt{#1}}}	% typesets as text in math mode

%\newcommand{\ts}[1]{\bftt{#1}} % ts = terminal symbol
% vphantom für trees http://tex.stackexchange.com/a/100859/6255
\newcommand{\ts}[1]{\bftt{#1}\protect\vphantom{\bftt{(}}} 

\allowdisplaybreaks[1] % split long equations on page breaks

% fix spacing for \left..\right
% http://tex.stackexchange.com/questions/2607/spacing-around-left-and-right
\let\originalleft\left
\let\originalright\right
\renewcommand{\left}{\mathopen{}\mathclose\bgroup\originalleft}
\renewcommand{\right}{\aftergroup\egroup\originalright}
%\usepackage{mleftright}

\theoremstyle{plain}
%\newtheorem{theorem}{Theorem}
\newtheorem{lemma}{Lemma}
\newtheorem{proposition}{Proposition}
\newtheorem*{corollary}{Corollary}

\declaretheoremstyle[
	spaceabove=10pt,
	notefont=\normalfont,
	headpunct={\normalfont:},
	bodyfont=\normalfont,
	postheadspace=\newline,
	qed=\ensuremath{\triangleleft}
]{mydefinition}

\declaretheoremstyle[
	spaceabove=0pt,
	spacebelow=0pt,
	notefont=\normalfont,
	headpunct={\normalfont:},
	bodyfont=\normalfont,
	postheadspace=\newline
]{myexample}

\declaretheoremstyle[
	spaceabove=10pt,
	notefont=\normalfont,
	headpunct={\normalfont:},
	bodyfont=\normalfont\itshape,
	postheadspace=\newline,
	qed=\ensuremath{\triangleleft}
]{mytheorem}

\declaretheoremstyle[
	numbered=no,
	spaceabove=6pt,
	headfont=\normalfont\itshape,
	notefont=\normalfont,
	headpunct={\normalfont:},
	bodyfont=\normalfont,
	qed=\ensuremath{\triangleleft}
]{myremark}

\renewcommand\thmcontinues[1]{Continued}

\declaretheorem[name=Definition,style=mydefinition]{definition}

\declaretheorem[name=Example,style=myexample,
	mdframed={
		hidealllines=true,leftline=true,bottomline=true,roundcorner=5pt,
    innertopmargin=0pt,
		rightmargin=0pt,innerrightmargin=0pt,
		innerlinewidth=0pt,middlelinewidth=0pt,
		outerlinecolor=shadecolor,outerlinewidth=3pt,
    backgroundcolor=none,
		skipabove=\topskip,skipbelow=\topskip,
		}
	]{example}

\declaretheorem[name=Theorem,style=mytheorem]{theorem}

\declaretheorem[name=Remark,style=myremark]{remark}
\declaretheorem[name=Note,style=myremark]{note}

%\newcommand*{\op}[1]{\operatorname{#1}}
\newcommand*{\op}[1]{\mathrm{#1}}

%\usepackage{stmaryrd} % for double brackets
\usepackage{textcomp} % for double brackets
%workaround until stmaryrd works
\newcommand*{\llbracket}{\text{\textlbrackdbl}}
\newcommand*{\rrbracket}{\text{\textrbrackdbl}}

\newcommand\concat{\ensuremath{\mathbin{+\mkern-5mu+}}}